\newpage
\renewcommand{\abstractname}{Реферат}
\begin{abstract}
	Данная работа состоит из введения, трех глав основной части, заключения и списка литературы. Написана на 30 листах.
	Содержит 23 иллюстрации, 4 таблицы и 3 листинга алгоритма. Список литературы содержит 19 наименований.

	Результатом работы является описание алгоритма, который позволяет генерировать все возможные простые связные
	альтернированные $k$-танглы с количеством перекрестков, не превосходящим заданное число. Основными его достоинствами
	по сравнению с аналогичными для узлов и зацеплений являются отсутствие необходимости в больших хеш-таблицах, малый по
	сравнению с общим количеством генерируемых объектов объем необходимой памяти со случайным доступом и малый объем
	выполняемой заведомо бесполезной работы.
	
	Также в работе описан алгоритм рисования диаграмм $k$-танглов.

	%\listoffigures
	%\listoftables
	%\listofalgorithms
\end{abstract}
